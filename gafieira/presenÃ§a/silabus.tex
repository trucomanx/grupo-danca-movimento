\documentclass[a4paper,10pt]{article}
%\documentclass[a4paper,10pt]{scrartcl}

\usepackage[utf8]{inputenc}

\title{Sílabus: Samba de Gafieira - Nível I}
\author{Fernando Pujaico Rivera}

\pdfinfo{%
  /Title    (Sílabus: Samba de Gafieira - Nível I )
  /Author   (Fernando Pujaico Rivera)
  /Creator  ()
  /Producer ()
  /Subject  ()
  /Keywords (Samba, Gafieira)
}

\begin{document}
\maketitle

\begin{table}[]
\begin{tabular}{|l|p{4cm}|p{6cm}|}
\hline
Semana & Tema & Descrição \\  \hline
1 &  Passos básicos: Frente e trás (F.T.) e Balanço & Serão estudados os movimentos de transição entre o balaço e F.T. e enfeites no giro do balanço.\\ \hline
2 &  Passo básico: Cruzado &  Serão estudados os movimentos de transição entre o balaço e o cruzado, e enfeites no cruzado. \\ \hline
3 &  Transição: Saída lateral e gancho. Postura: X &  Serão estudados os movimentos de transição entre F.T. e cruzado, além das posturas notáveis e enfeites na saída lateral.\\ \hline
4 &  Passo: Gatilho &  Serão estudados os movimentos de transição entre F.T. e cruzado. \\ \hline
5 &  Passo: Caminhada em contratempo & Será estudado a distribuição de tempos na caminhada em contratempo. \\ \hline
5 &  Passo: Facão, Postura: Facão & Será estudado vários estilos de facão (movimento discreto e ondulado) e a postura de finalização. \\ \hline
\end{tabular}
\end{table}

\end{document}
